\section{Implementation}

\subsection{Overview}

Our framework consists of 4 components (Table \ref{table:componments}),
Each of those componment is designed to be replaceable, and they interact with
other components(Fig.\ref{figure:aoss}) with XML-RPC based protocol exactly as
what ROS does.

\begin{table}[H]
  \renewcommand{\arraystretch}{1.3}
  \caption{Componments of the framework}
  \label{table:componments}
  \centering
  \begin{tabular}{ll}
     \hline
     \bfseries Name & \bfseries Purpose \\
     \hline
     Illuminate & Extented version of ROS Master \\ 
     Daemon     & In-container daemon \\
     Shadow     & Client-side utilities \\
     Middleman  & Handles address mapping from container to host \\
     \hline
  \end{tabular}
\end{table} 

To implement above features, the biggest problem is manitaining multiple active
instance of a specified ROS Service at same time.

ROS Master used a thread-safe dictionary to manitain the name-node mapping,
which does not allowing register multiple instance of ROS Service with same name.
Whenever a name crash occurs, only the latest registered instance is
accessable via Master's \emph{lookupService} API call.

We extented the original ROS Master to overcome this
limitation(Fig.\ref{figure:aoil}), and named this variant as \emph{Illuminate}.

\begin{figure}[!t]
\centering
\includegraphics[width = \figwidth]{figures-1.pdf}
\caption{Interaction Between Componments}
\label{figure:aoss}
\end{figure}

Instances are no longger ordinary \emph{ROS Nodes} in the \emph{ROS Graph},
but \emph{Cells} in their containers.

For clients of \emph{Illuminate}, as the only different between \emph{Master} and \emph{Illuminate}
is that the latter's Service register handling logic treats name crashes as normal requests, 
and built with load-balancing support.

While staying the protocol not modified, the original cline-side ROS
installations just work as usual without any extra configuration or modification.

However, for feature \ref{feature:fallback}, the traditional way of connecting current machine into ROS Graph,
setting up environment variable \emph{ROS\_MASTER\_API}), cannot help any more.
That's the direct motivatoin of introducing \emph{Shadow}.
Shadow provides client-side support of Feature \ref{feature:mul}, \ref{feature:fallback}. 

\emph{Deamon} and \emph{Middleman} are implementation-related componment providing distributing support.
They watch, control the running of ROS Services, and provide exposed address to Illuminate.

\subsection{Containerizing ROS Service}
Both traditional virtualization and containerization(or sandbox) provides enough ablity to distribute any existing ROS Service.
Comparing with virtualization, containerization brings much less overhead.

Docker, which is one of the most popular lightweight containerizing solution offically supported by ROS.
This method is developed mainly based on Docker, but you can use it atop of many solutions with a tiny adapter. 

\subsection{Overriding ROS Master API}
The main techique used in tweaking these XML-RPC based APIs is proxying.
Within this extra intermediate, we intercept interested API calls,
and route the others to appropriate endpoint (see TABLE \ref{table:interested-apis}).

Generally, we intercept all API about Services so that we can modify the default behavior of the register,
unregister and lookup of a Service.
However, for different componments, interested API can be different (Table \ref{table:int}).
These implementation related details are discussed in subsections for componment.
 
\begin{table}
    \renewcommand{\arraystretch}{1.3}
    \caption{Interested APIs in Different Componment}
    \label{table:interested-apis}
    \centering
    \begin{tabular}{|l|l|l|}
        \hline
        \bfseries Componment & \bfseries  API & \bfseries Purpose \\
        \hline
        \multirow{1}{*}{Illuminate} & lookupService & generalized lookup \\
        \hline
        \multirow{6}{*}{Daemon} & registerService & expose a Cell to Illuminate\\
        \cline{2-3} & unregisterService & unregister current Cell \\
        \cline{2-3} & registerPublisher & \multirow{4}{*}{Simply intercept them} \\
        \cline{2-2} & registerSubscriber    &  \\
        \cline{2-2} & unregisterPublisher   &  \\
        \cline{2-2} & unregisterSubscriber  &  \\
        \hline
        \multirow{1}{*}{Shadow} & lookupService & generalized lookup \\
        \hline
    \end{tabular}
\end{table} 

\subsection{Illuminate Details}

\begin{figure}[!t]
\centering
\includegraphics[width = \figwidth]{figures-3.pdf}
\caption{Architecture of Illuminate}
\label{figure:aoil}
\end{figure}

Illuminate creates a namespace to maintain all the Cells independent of
original ROS Services.

To maximum the compatiblity, Cells have lower priority than Services.

Illuminate owns all functionality of ROS Master,
extented it with 3 new APIs(Table \ref{table:illuminate-api}),
and adjusts the \emph{lookupService} function's logic.

All the 3 new APIs have exactly the same arguments and
return value like the Service version (Table \ref{table:illuminate-api}).

\begin{table}[H]
  \caption{New API in Illuminate}
  \label{table:illuminate-api}
  \centering
  \begin{tabular}{ll}
    \hline
    \bfseries Name & \bfseries Purpose \\
    \hline
    registerCell   & register Cells into registeration manager  \\
    unregisterCell & unregister Cells from registeration manager \\
    lookupCell     & lookup Cells, not exposed \\
    \hline
  \end{tabular}
\end{table} 


\subsubsection{Routines}
There're totally 3 possible execute routines for each valid API call.

\paragraph{Service Lookup}
Service lookup shares same logic with both client-side and daemon-side requests.
Like the red pathes shown in Fig.\ref{figure:aoil}, Illuminate will first try 
to find a appropriate service. If none of global services matches, 
Illuminate try to find a appropriate Cell.
If nothing found again, respond a failure as final response.
 
\paragraph{Cells}
Like the blue pathes shown in Fig.\ref{figure:aoil}, explicit Cell operations
are directly handled by Illuminate.

\paragraph{Non-interested API Calls}
Like the red pathes shown in Fig.\ref{figure:aoil},
Illuminate simply transmit non-interested to embeded Master code.

\subsubsection{Load Balancing}
Implementing load-balancing algorithms based on \emph{the real payload of each Cell} is overcomplicated.

To minimize the modification of original ROS, we decided to use the \emph{LRU} algorithm to detected which cell should be returned for current request, just like \emph{Nginx}'s default behavior.


\subsection{Daemon and Middleman Details}

\begin{figure}[!t]
\centering
\includegraphics[width = \figwidth]{figures-5.pdf}
\caption{Architecture of Deamon and Middleman}
\label{figure:aodm}
\end{figure}

Deamon runs on each active container.
Its mainly function is handling Cells (un)register requests.
\begin{table}[H]
  \renewcommand{\arraystretch}{1.3}
  \caption{Deamon's behavior}
  \label{table:demans-behavior}
  \centering
  \begin{tabular}{|l|l|l|}
     \hline
     \bfseries \# & \bfseries API & \bfseries Behaviour \\
     \hline
     1 & registerService & expose current Service to Illuminate\\
     \hline
     2 & unregisterService & unregister current Service instance \\
     \hline
     3 & registerPublisher & \multirow{4}{*}{Simply intercept them} \\
     \cline{1-2}
     4 & registerSubscriber    &  \\
     \cline{1-2}
     5 & unregisterPublisher   &  \\
     \cline{1-2}
     6 & unregisterSubscriber  &  \\
     \hline
     7 & all the others & transmit to Illuminate \\
     \hline
  \end{tabular}
\end{table}

Outside container, Middleman handles container IP - host port mapping.

While each independent ROS Service will listen on random port on its
initializing process, transcendental method would not works any way. 
So we tweaked it a bit in each container.
Just like shown in Fig.\ref{figure:aodm}, we mapped this random port to
port 30000 using iptables, and expose port 30000 to Docker Hosts.

Notice that, current Docker releases has not included an introspect mechanism.
Engage the access from inside the Container to host is not a secure practice.

So we used a three-stage solution with a extra container running a Key-Value storage componment(Radis for this case).
\begin{enumerate}
    \item Register the mapped port to Key-Value storage right after container was created.
    \item Expose port 30000 for each Cells' container
    \item Redirect income requests from 30000 to listening port of the Service right after its initialization
\end{enumerate}

\subsubsection{Routines}
\paragraph{Launch}
For each Service in container, all the requests will be sent to Deamon,
which is shown in Fig.\ref{figure:aodm} (a).
For \emph{registerService} API calls, Deamon modifys the iptable rules to
map the current launching Service's listening port to port 30000 of the container.
Then, a query request is sent to Middleman, as action (b).

\paragraph{Address Reassembling}
Middleman requires the containers information from Docker (c),
pick the binded port, and host's IP, assemble an accessable address and
return to Daemon. 

\paragraph{Register}
In action (d), the reassembled address will replace the Servic emitted one,
and sent to Illuminate.
The Service is finally launched and registered. API calls besides
\emph{lookupService} is processed as list in TABLE \ref{table:interested-apis}.

\subsection{Shadow Details}

\begin{figure}[!t]
\centering
\includegraphics[width = \figwidth]{figures-7.pdf}
\caption{Architecture of Shadow}
\label{figure:aosh}
\end{figure}

Finally, robots need clinet-side infrastructure to access Cells,
in the Illuminate managed, independent namespace.

Using the same techique of creating \emph{Daemon}, we created a lightweight
proxy of Illuminate, or ROS Master on client side, called \emph{Shadow}.

For different use cases, both original and customized client programs
works(TABLE \ref{table:shadow-compatible}).

\begin{table}
  \caption{Client-side ROS Compatiblity}
  \label{table:shadow-compatible}
  \centering
  \begin{tabular}{ll}
    \hline
    \bfseries Feature & \bfseries Origin ROS Compatiblity \\
    \hline
    Access Cells    & Yes \\
    Private Service & No  \\
    Fallback        & No  \\
    \hline
  \end{tabular}
\end{table} 

\subsubsection{Accessing Cells}
As Services have higher priority than Cells, the \emph{lookupService} will
work, with or with out Shadow.

In Fig.\ref{figure:aosh}, lookuping private Service is marked as routine \emph{ab},
lookuping Cells or Services on the ROS Graph is marked as routine \emph{ac}.
Particular combination of routines used in a single request is decided by
associated packages' configuration.
 
\subsubsection{Supporting private Nodes}
Just like Deamon intercept some register actions to keep only the Service itself is
correctly expose to Illuminate, Shadow intercepts all these requests and process them
on client-side, as routine \emph{de}.

If not explicitly specified, a Service is registered as public one,
to keep compatible with the original ROS's behaviors(routine \emph{df}).

\subsubsection{Supporting Fallback}
Fallback only matters which combination of lookup actions will be used.

When a package is directly or recursively marked as private, Shadows' lookup 
process only checks local Services, and all the (un)register is processed locally.

All the APIs besides \emph{lookupService} will be transmit to Illuminate, 
as routine \emph{gh}.

\subsubsection{Configuration}
To discrbe a Service's usage and its lookup behavior in
a client's perspective, we defined 3 properties, priority,
fallback and recursive(see table \ref{table:srv-opt}).

\begin{table}
  \renewcommand{\arraystretch}{1.3}
  \caption{Options to configure a ROS Service}
  \label{table:srv-opt}
  \centering
  \begin{tabular}{|l|l|l|}
    \hline
    \bfseries Name & \bfseries Possible values & \bfseries Meaning \\
    \hline
    \multirow{2}{*}{priority}  & Local  & use local Nondes first \\
    \cline{2-3}
                               & Remote & use remote Cells first \\
    \hline
    \multirow{3}{*}{fallback}  & True   & allowing fallback \\
    \cline{2-3}
                               & False  & disallowing fallback \\
    \cline{2-3}
                               & non-empty list of services & try  \\
    \hline
    \multirow{2}{*}{recursive} & True   & apply above rules to dependency \\
    \cline{2-3}
                               & False  & do not apply to dependency \\
    \hline
  \end{tabular}
\end{table} 

Both JSON or YAML is accpted.
