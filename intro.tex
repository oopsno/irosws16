\section{Introduction}

\subsection{Motivation}

There are many existing ROS Services works very well in its specified domain.
However, original ROS does not allow running multiple Service instances with
same name at same time. 

ROS client implementation brings with multi-threading support. This works fine
in samll-scale scenario, but as the ROS Graph grows, multi-threading exposes
three mainly problem:

\begin{itemize}
    \item Execution is limited to single physical machine
    \item No efficient management mechanism
\end{itemize}

These goes serious as the system payload scales, and causes higher mean response time.

To overcome these shortcommings, and, provide a easy-to-deploy solution with 
ROS compatibility, we advanced a method to deploy ROS Services on cluster based
on containerization techique and a series of extensions of ROS Graph's XML-RPC based
communication.

\subsection{Feature}

Key-points of building this system is

\begin{itemize}
    \item Use the original \emph{ROS Service} as a model to provide remote calculation
    \item Allowing distributed deployment
    \item Provide code-level compatibility
\end{itemize}

\subsection{Limitation}

As ROS promises Services are singleton, authors focus on original ROS may ignores some
synchronization mechanisms.
In containerization environment, each instance of Services runs in independent virtual
filesystem, so wound not crash, but not synchronized. This may causes undefined behaviours.

\subsection{Problem definition}

Develop a lightweight, code-level compatible, Docker based distributed technology of running ROS Services, which proxy following features:

\begin{enumerate}
    \item Distributing multiple ROS Service instance \label{feature:mul}
    \item Fallback supporting \label{feature:fallback}
    \item ROS compatibility \label{feature:comp}
\end{enumerate}


